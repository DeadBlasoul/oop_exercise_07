\documentclass[12pt]{article}

\usepackage{fullpage}
\usepackage{multicol,multirow}
\usepackage{tabularx}
\usepackage{ulem}
\usepackage[utf8]{inputenc}
\usepackage[russian]{babel}
\usepackage{minted}

\usepackage{color} %% это для отображения цвета в коде
\usepackage{listings} %% собственно, это и есть пакет listings

\lstset{ %
language=C,                 % выбор языка для подсветки (здесь это С)
basicstyle=\small\sffamily, % размер и начертание шрифта для подсветки кода
numbers=left,               % где поставить нумерацию строк (слева\справа)
%numberstyle=\tiny,           % размер шрифта для номеров строк
stepnumber=1,                   % размер шага между двумя номерами строк
numbersep=5pt,                % как далеко отстоят номера строк от подсвечиваемого кода
backgroundcolor=\color{white}, % цвет фона подсветки - используем \usepackage{color}
showspaces=false,            % показывать или нет пробелы специальными отступами
showstringspaces=false,      % показывать или нет пробелы в строках
showtabs=false,             % показывать или нет табуляцию в строках
frame=single,              % рисовать рамку вокруг кода
tabsize=2,                 % размер табуляции по умолчанию равен 2 пробелам
captionpos=t,              % позиция заголовка вверху [t] или внизу [b] 
breaklines=true,           % автоматически переносить строки (да\нет)
breakatwhitespace=false, % переносить строки только если есть пробел
escapeinside={\%*}{*)}   % если нужно добавить комментарии в коде
}


\begin{document}
\begin{titlepage}
\begin{center}
\textbf{МИНИСТЕРСТВО ОБРАЗОВАНИЯ И НАУКИ РОССИЙСОЙ ФЕДЕРАЦИИ
\medskip
МОСКОВСКИЙ АВЦИАЦИОННЫЙ ИНСТИТУТ
(НАЦИОНАЛЬНЫЙ ИССЛЕДОВАТЬЕЛЬСКИЙ УНИВЕРСТИТЕТ)
\vfill\vfill
{\Huge ЛАБОРАТОРНАЯ РАБОТА №7} \\
по курсу объектно-ориентированное программирование
I семестр, 2019/20 уч. год}
\end{center}
\vfill

Студент \uline{\it {Попов Данила Андреевич, группа М8О-208Б-18}\hfill}

Преподаватель \uline{\it {Журавлёв Андрей Андреевич}\hfill}

\vfill
\end{titlepage}

\subsection*{Условие}

Спроектировать графический редактор с графическим интерфейсом.

Редактор должен соответствовать следующему функционалу:
\begin{enumerate}
\item создание нового документа 
\item импорт документа из файла
\item экспорт документа в файл
\item создание/удаление графических примитивов 
\item отображение документа на экране
\item операция undo
\end{enumerate}

\subsection*{Описание программы}

Исходный код лежит в 8 файлах:
\begin{enumerate}
\item app/main.cpp: точка входа в программу
\item src/application.cpp: определение центрального класса
\item src/application.hpp: объявление центрального класса
\item src/builders.hpp: обработчики пользовательского ввода
\item src/figures.hpp: определение графических примитивов
\item src/editor/brush.hpp: определение кисти
\item src/editor/builder.hpp: определение интерфейса обработчика пользовательского ввода
\item src/editor/drawable.hpp: определение интерфейса графического примитива
\item src/editor/figure.hpp: определение интерфейса графической фигуры
\item src/editor/storage.hpp: хранилище текущего состояния канваса 
\item src/geom/algorithm.hpp
\item src/geom/point.hpp
\item src/geom/polygon.hpp
\item src/system/application\_base.cpp: определение базового класса приложения
\item src/system/application\_base.hpp: объявление базового класса приложения
\item src/system/renderer.hpp: класс отрисовщика
\item src/system/sdl2.cpp: sdl2 C++ wrapper 
\item src/system/sdl2.hpp

\end{enumerate}

\subsection*{Дневник отладки}

Не смог осилить выпадающие окошки. Сохранение и открытие документов происходит через командную строку. 

\subsection*{Недочёты}

Сериализатор до жути тривиальный.

\subsection*{Выводы}

ImGui как панацея от всех болезней. 

\vfill

\subsection*{Исходный код}


{\Huge main.cpp}
\inputminted
    {C++}{app/main.cpp}
    \pagebreak

{\Huge src/application.cpp}
\inputminted
    {C++}{src/application.cpp}
    \pagebreak

{\Huge src/application.hpp}
\inputminted
    {C++}{src/application.hpp}
    \pagebreak

{\Huge src/builders.hpp}
\inputminted
    {C++}{src/builders.hpp}
    \pagebreak

{\Huge src/figures.hpp}
\inputminted
    {C++}{src/figures.hpp}
    \pagebreak

{\Huge src/editor/brush.hpp}
\inputminted
    {C++}{src/editor/brush.hpp}
    \pagebreak

{\Huge src/editor/builder.hpp}
\inputminted
    {C++}{src/editor/builder.hpp}
    \pagebreak

{\Huge src/editor/drawable.hpp}
\inputminted
    {C++}{src/editor/drawable.hpp}
    \pagebreak

{\Huge src/editor/figure.hpp}
\inputminted
    {C++}{src/editor/figure.hpp}
    \pagebreak

{\Huge src/editor/storage.hpp}
\inputminted
    {C++}{src/editor/storage.hpp}
    \pagebreak

{\Huge src/geom/algorithm.hpp}
\inputminted
    {C++}{src/geom/algorithm.hpp}
    \pagebreak

{\Huge src/geom/point.hpp}
\inputminted
    {C++}{src/geom/point.hpp}
    \pagebreak

{\Huge src/geom/polygon.hpp}
\inputminted
    {C++}{src/geom/polygon.hpp}
    \pagebreak

{\Huge src/system/application\_base.cpp}
\inputminted
    {C++}{src/system/application_base.cpp}
    \pagebreak

{\Huge src/system/application\_base.hpp}
\inputminted
    {C++}{src/system/application_base.hpp}
    \pagebreak

{\Huge src/system/renderer.hpp}
\inputminted
    {C++}{src/system/renderer.hpp}
    \pagebreak

{\Huge src/system/sdl2.cpp}
\inputminted
    {C++}{src/system/sdl2.cpp}
    \pagebreak

{\Huge src/system/sdl2.hpp}
\inputminted
    {C++}{src/system/sdl2.hpp}
    \pagebreak
    
\end{document}
